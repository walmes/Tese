A água é indispensável para produção das culturas pois está envolvida
no transporte de nutrientes, reações químicas, processos físicos e
manutenção da vida do solo. O conhecimento sobre a curva de rentenção
de água (CRA) do solo é fundamental para estabelecer estratégias de
manejo. A qualidade física do solo é depende da CRA e os parâmetros
$I$, tensão do ponto de inflexão da CRA, e $S$, taxa de variação no
ponto de inflexão, considerados como indicadores da qualidade física,
são parâmetros relacionados a medidas descritivas da distribuição do
tamanho de poros do solo. Com este trabalho, objetiva-se verificar o
efeito da posição de amostragem e profundidade do solo sobre os
parâmetros $I$ e $S$ da CRA. Para isso 1) considerou-se ANOVA simples
e 2) ANOVA ponderada pela variância das estimativas desses parâmetros
em cada unidade experimental em comparação com 3) o uso de modelos não
lineares de efeito misto em uma parametrização desenvolvida para $I$ e
$S$. Nenhum dos métodos alternativos de análise foi superior ao modelo
não linear de efeitos mistos na parametrização desenvolvida, que
apresentou intervalos mais estreitos para estimativas dos parâmetros e
apontou efeito de posição e profundidade de coleta nos parâmetros $I$
e $S$.\\
\newline
\noindent {Palavras-chave}: Reparametrização. Função de parâmetros. Método delta. Curva de retenção de água.
