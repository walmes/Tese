\section{INTRODUÇÃO}

A ideia básica da regressão não linear é a mesma da regressão linear:
relacionar uma resposta $Y$ com um vetor de variáveis preditoras
$\mathbf{x} = (x_1,\ldots,x_k)^\top$. Os modelos de regressão não
linear são caracterizados pelo fato de a função de predição depender
não linearmente de algum dos parâmetros. Embora não necessariamente, a
regressão linear é usada para especificação de modelos puramente
empíricos, enquanto que os modelos de regressão não linear são
considerados quando existe algum conhecimento prévio para sustentar
que a relação entre resposta e preditores segue uma particular forma
funcional. Tal conhecimento pode ser desde uma equação diferencial que
remete à particular modelo, como é o caso de modelos de crescimento,
ou simplesmente uma restrição sobre a função, como o de a função ser
monótona, típico de curvas de acúmulo, para a qual pode-se ter várias
funções disponíveis.

Uma das principais vantagens do modelo de regressão não linear, é que
frequentemente existe interpretação para a maioria de seus parâmetros
\cite{Schabenberger2002}.  Esses parâmetros então passam ser o foco da
investigação que, na sua forma mais simples, consiste em determinar
intervalos de confiança e testar hipóteses. No entanto, uma situação
comum é a necessidade de fazer inferência sobre uma função dos
parâmetros \cite{Bender1996}.  Um exemplo simples é a equação de
segundo grau $f(x) = \beta_0 + \beta_1 x + \beta_2 x^2$, que é um
modelo linear no qual o ponto crítico $x_c = -\beta_1/(2\beta_2)$ é
alvo de inferência em situações de otimização de processos
\cite{Bas2007}. Uma vez estimados seus parâmetros, inferência sobre
$x_c$ pode ser feita pelo método delta, por simulação Monte Carlo ou
por métodos \emph{bootstrap} \cite{Seber2003}. Embora tais
procedimentos permitam obter intervalos de confiança e conduzir testes
de hipótese, existem ainda outras formas vantajosas de inferir ou
modelar o parâmetro que são as extensões ligadas aos modelos não
lineares de efeitos mistos \cite{Pinheiro2009} e a inferência
bayesiana \cite{Denison2002}.

\newpage
\section{REPARAMETRIZAÇÃO}

Considere um modelo não linear
\begin{equation}\label{modelonaolinear}
f(\mathbf{x}, \boldsymbol{\theta})
\end{equation}
em que $f$ é uma função não linear que depende do vetor de covariáveis
$\mathbf{x}$ e $\boldsymbol{\theta}$ é seu vetor de $p$
parâmetros. Seja $\vartheta = g(\boldsymbol{\theta})$ o parâmetro de
interesse em que $g$ é uma função monótona e diferenciável em relação
à $\boldsymbol{\theta}$.  O objetivo com a reparametrização é fazer
com que $\vartheta$ seja um elemento do vetor de parâmetros do
modelo. Isso é obtido por substituição de algum dos $p$ elementos de
$\boldsymbol{\theta}$ por $\vartheta$.  Para isso, sistematizou-se o
procedimento em três etapas:
\begin{enumerate}
\item Expressar o parâmetro de interesse como função dos elementos de
  $\boldsymbol{\theta}$, ou seja, $\vartheta =
  g(\boldsymbol{\theta})$;
\item Escolher um dos elementos $\theta_i$ de $\boldsymbol{\theta} =
  (\theta_i, \boldsymbol{\theta}_{-i})$ para ser colocado em função de
  $\vartheta$ de tal forma a obter $\theta_i =
  h(\boldsymbol{\theta}_{-i}, \vartheta)$;
\item Substituir $\theta_i$ em (\ref{modelonaolinear}) pela expressão
  obtida no passo anterior, $h(\boldsymbol{\theta}_{-i}, \vartheta)$,
  fazendo as simplificações convenientes. Assim o modelo
  (\ref{modelonaolinear}) pode ser expresso como
$$f(\mathbf{x}, \boldsymbol{\theta}_{-i}, \vartheta)$$.
\end{enumerate}
A função $h$ é a inversa de $g$ em $\theta_i$.

No passo 2 recomenda-se priorizar aquele elemento de
$\boldsymbol{\theta}$ com menor

\subsection{Reparametrização 1:1 - Modelos para acúmulo com ênfase na fração do total}

O modelo Michaelis-Menten foi inicialmente proposto para para
descrever a cinética de reações químicas \cite{Michaelis1913}. Tal
modelo envolve uma função monótona crescente côncava a partir da
origem.  Atualmente observa-se a aplicação desse modelo em diversos
contextos; um deles é a descrição do acúmulo de potássio liberado do
solo \cite{Zeviani2012}. Sua forma funcional é
\begin{equation}
f(x; \theta_a, \theta_v) = \frac{\theta_a x}{\theta_v+x}, \qquad
x\geq0\,\, (\text{X}),
\end{equation}
em que $\theta_a\geq 0$, é a assintota superior (Y) e representa o
conteúdo total de nutriente liberado, e $\theta_v> 0$ é tempo de meia
vida (X) ou tempo para fração meio.

\subsection{Outros modelos}

Além dos modelos considerados para exemplificar o procedimento de
reparametrização, outros modelos frequentemente aplicados em Ciências
Agrárias foram reparametrizados e estão apresentados nas tabelas de
\ref{tab:catalog1} à \ref{tab:catalog3}. A descrição de cada modelo,
em termos de propriedades da função não linear, interpretação dos
parâmetros é fornecida a seguir em uma lista numerada de acordo com as
linhas das tabelas de \ref{tab:catalog1} a \ref{tab:catalog3}.

\def\captabelas{Reparametrizações desenvolvidas com ênfase na
  interpretação dos parâmetros de modelos de regressão não linear
  aplicados em Ciências Agrárias}

\begin{sidewaystable}
 \caption{\captabelas}\label{tab:catalog1}
\begin{small}
 % A TABELA TEM QUE TER 19 CM DE COLUNA NO, E 12.6 CM DE LINHAS
%\begin{tabular}{|c|c|c|c|c|}
%\begin{tabular*}{\textwidth}{@{\extracolsep{\fill}}|c|c|c|c|c|}
%\begin{tabular}{|C{0.4cm}|C{5.3cm}|C{2.7cm}|C{2.7cm}|C{5.5cm}|} % ESSA SOMA TEM QUE DAR 16.6
%\begin{tabular}{C{0.4cm}C{5.3cm}C{2.7cm}C{2.7cm}C{5.5cm}} % ESSA SOMA TEM QUE DAR 16.6
\begin{tabular*}{\textwidth}{@{\extracolsep{\fill}}ccccc}
\hline 
id &
Modelo original & $\vartheta = g(\boldsymbol{\theta})$ & $\theta_i = g^{-1}(\vartheta, \boldsymbol{\theta}_{-i})$ & Modelo reparametrizado\tabularnewline
\hline 
%---------------------------------------------------------------------------------------------------
% michaelis menten, monod, hiperbola de 2 parametros
1 &
$\displaystyle\frac{\theta_a x}{\theta_v+x}$ &
$\displaystyle\vartheta_q = \theta_v\left(\frac{q}{1-q}\right)$ &
$\displaystyle\theta_v = \vartheta_q\left(\frac{1-q}{q}\right)$ &
$\displaystyle\frac{\theta_a x}{\vartheta_q\left(\frac{1-q}{q}\right)+x}$\tabularnewline
%\hline 
%---------------------------------------------------------------------------------------------------
% modelo michaelis aumentado crescente
2 &
$\displaystyle\frac{\theta_a}{1+\left(\frac{\theta_v}{x}\right)^{\theta_c}}$ &
$\displaystyle\vartheta_q = \theta_v\left(\frac{1-q}{q}\right)^{-1/\theta_c}$ &
$\displaystyle\theta_v = \vartheta_q\left(\frac{1-q}{q}\right)^{1/\theta_c} $ &
$\displaystyle\frac{\theta_a}{1+\frac{1-q}{q}\left(\frac{\vartheta_q}{x}\right)^{\theta_c}}$\tabularnewline
%\hline 
%---------------------------------------------------------------------------------------------------
% modelo michaelis aumentado decrescente
3 &
$\displaystyle\frac{\theta_a}{1+\left(\frac{x}{\theta_v}\right)^{\theta_c}}$ &
$\displaystyle\vartheta_q = \theta_v\left(\frac{1-q}{q}\right)^{1/\theta_c}$ &
$\displaystyle\theta_v = \vartheta_q\left(\frac{1-q}{q}\right)^{-1/\theta_c} $ &
$\displaystyle\frac{\theta_a}{1+\frac{1-q}{q}\left(\frac{x}{\vartheta_q}\right)^{\theta_c}}$\tabularnewline
%\hline 
%---------------------------------------------------------------------------------------------------
% modelo michaelis aumentado de novo
4 &
$\displaystyle\frac{\theta_a x^{\theta_c}}{\theta_v+x^{\theta_c}}$ &
$\displaystyle\vartheta_q = \left(\frac{\theta_v q}{1-q}\right)^{1/\theta_c}$ &
$\displaystyle\theta_v = \vartheta_q^{\theta_c} \frac{1-q}{q}$ &
$\displaystyle\frac{\theta_a x^{\theta_c}}{\vartheta_q\left(\frac{1-q}{q}\right)+x^{\theta_c}}$\tabularnewline
%\hline 
%---------------------------------------------------------------------------------------------------
% modelo exponencial assintótico
5 &
$\theta_a(1-\exp\{-\theta_c x\})$ &
$\vartheta_q = \displaystyle-\frac{\log(1-q)}{\theta_c}$ &
$\theta_c = \displaystyle-\frac{\log(1-q)}{\vartheta_q}$ &
$\theta_a(1-\exp\{x \log(1-q)/\vartheta_q\})$\tabularnewline
%\hline
%---------------------------------------------------------------------------------------------------
% modelo exponencial assintótico para tempo de aquecimento
6 &
$\begin{cases} \theta_a(1-\exp\{-\theta_1(x-\theta_0)\}) &,\, x\geq \theta_0 \\  0 &,\, x<\theta_0  \end{cases}$ &
$\displaystyle\vartheta_q = \frac{\log(1-q)}{\theta_1}+\theta_0$ &
$\displaystyle\theta_1 = \frac{\log(1-q)}{\vartheta_q-\theta_0}$ &
$\displaystyle \theta_a\left(1-\exp\left\{\log(1-q)\left(\frac{x-\theta_0}{\vartheta_q-\theta_0}\right)\right\}\right)$\tabularnewline
%\hline 
%---------------------------------------------------------------------------------------------------
% modelo potência para dose econômica
7 &
$\theta_0-\theta_1 x^{\theta_2}$ &
$\vartheta_q = \displaystyle\frac{q}{\theta_1}^{1/\theta_2}$ &
$\displaystyle\theta_2 = \frac{\log(q)-\log(\theta_1)}{\log(\vartheta_q)} $ &
$\displaystyle\theta_0-\theta_1 x^{\frac{\log(q)-\log(\theta_1)}{\log(\vartheta_q)}}$\tabularnewline
%\hline 
%---------------------------------------------------------------------------------------------------
% modelo assintótico, também conhecido como (A+B)-B*C^x ou A-exp(-B)*C^x
8 &
$\theta_0+\theta_1(1-\theta_c^x)$ &
$\displaystyle\vartheta_q = \frac{\log(1+q/\theta_1)}{\log(\theta_c)}$ &
$\displaystyle\theta_1 = -\frac{q}{1-\theta_c^{\vartheta_q}}$ &
$\displaystyle\theta_0-q \left(\frac{1-\theta_c^x}{1-\theta_c^{\vartheta_q}}\right)$\tabularnewline
\hline 
%\end{tabular}
\end{tabular*}


\end{small}
\end{sidewaystable}

\begin{sidewaystable}
 \caption{(cont.) \captabelas.}\label{tab:catalog2}
\begin{small}
 %\begin{tabular}{|c|c|c|c|c|}
%\begin{tabular*}{\textwidth}{@{\extracolsep{\fill}}|c|c|c|c|c|}
%\begin{tabular}{|C{0.4cm}|C{4.7cm}|C{3.4cm}|C{3.4cm}|C{4.7cm}|} % ESSA SOMA TEM QUE DAR 16.6
%\begin{tabular}{C{0.4cm}C{4.7cm}C{3.4cm}C{3.4cm}C{4.7cm}} % ESSA SOMA TEM QUE DAR 16.6
\begin{tabular*}{\textwidth}{@{\extracolsep{\fill}}ccccc}
\hline 
id &
Modelo original & $\vartheta = g(\boldsymbol{\theta})$ & $\theta_i = g^{-1}(\vartheta, \boldsymbol{\theta}_{-i})$ & Modelo reparametrizado\tabularnewline
\hline 
%---------------------------------------------------------------------------------------------------
% modelo plato-linear
9 &
$\begin{cases} \theta_0+\theta_1 x &,\, x\leq \theta_b \\  \theta_0+\theta_1 \theta_b &,\, x>\theta_b  \end{cases}$ &
$\vartheta_b = \theta_0+\theta_1 \theta_b$ &
$\theta_0 = \vartheta_b-\theta_1 \theta_b$ &
$\begin{cases} \vartheta_b+\theta_1(x-\theta_b) &,\, x\leq \theta_b \\ \vartheta_b &,\, x>\theta_b \end{cases}$\tabularnewline
%\hline
%---------------------------------------------------------------------------------------------------
% modelo linear segmentado
10 &
$\begin{cases} \theta_0+\theta_1 x &,\, x\leq \theta_b \\ \theta_0+\theta_1 \theta_b+\theta_2(x-\theta_b) &,\, x> \theta_b \end{cases}$ &
$\vartheta_b = \theta_0+\theta_1 \theta_b$ &
$\theta_0 = \vartheta_b-\theta_1 \theta_b$ &
$\begin{cases} \vartheta_b+\theta_1(x-\theta_b) &,\, x\leq \theta_b \\ \vartheta_b+\theta_2(x-\theta_b) &,\, x> \theta_b \end{cases} $\tabularnewline
%\hline
%---------------------------------------------------------------------------------------------------
% modelo quadrático
\multirow{2}{*}{11} &
\multirow{2}{*}{$\theta_0+\theta_1 x+\theta_2 x^2$} &
$\displaystyle\vartheta_x = -\frac{\theta_1}{2\theta_2}$ &
$\theta_1 = 2 \theta_2 \vartheta_x$ &
\multirow{2}{*}{$\vartheta_y+\theta_2(x-\vartheta_x)^2$}\tabularnewline
 & 
 &
$\vartheta_y = \theta_0+\theta_1\vartheta_x+\theta_2\vartheta_x^2$ &
$\theta_0 = \vartheta_y - \theta_1\vartheta_x - \theta_2\vartheta_x^2$
 & \tabularnewline
%\hline
%---------------------------------------------------------------------------------------------------
% modelo quadrático-plato
\multirow{4}{*}{12} &
%\multirow{2}{*}{$\begin{cases} \theta_0+\theta_1 x+\theta_2 x^2 &,\, x\leq -\theta_1/(2\theta_2)\\ \theta_0+\theta_1\left(\frac{-\theta_1}{2\theta_2}\right)+%%\theta_2\left(\frac{-\theta_1}{2\theta_2} \right)^2 &,\, x> -\theta_1/(2\theta_2) \end{cases}$} &
\multirow{4}{*}{$\begin{cases} \theta_0+\theta_1 x+\theta_2 x^2, \\ \qquad x\leq -\theta_1/(2\theta_2)\\ \theta_0+\theta_1\left(\frac{-\theta_1}{2\theta_2}\right)+\theta_2\left(\frac{-\theta_1}{2\theta_2} \right)^2, \\ \qquad  x> -\theta_1/(2\theta_2) \end{cases}$} &
$\displaystyle\vartheta_x = -\frac{\theta_1}{2\theta_2}$ &
$\theta_1 = 2 \theta_2 \vartheta_x$ &
\multirow{4}{*}{$\begin{cases} \vartheta_y+\theta_2(x-\vartheta_x)^2 &,\, x\leq \vartheta_x\\ \vartheta_y &,\, x> \vartheta_x \end{cases}$}\tabularnewline
 &
 &
$\vartheta_y = \theta_0+\theta_1\vartheta_x+\theta_2\vartheta_x^2$ &
$\theta_0 = \vartheta_y - \theta_1\vartheta_x - \theta_2\vartheta_x^2$
 & \tabularnewline
 & & & & \tabularnewline
 & & & & \tabularnewline
 & & & & \tabularnewline
%\hline 
%---------------------------------------------------------------------------------------------------
% modelo produção competição (Bleasdale & Nelder, 1960)
\multirow{2}{*}{13} &
\multirow{2}{*}{$x(\theta_0+\theta_1 x)^{-1/\theta_2}$} &
$\displaystyle\vartheta_x = \frac{\theta_0}{\theta_1}\left(\frac{\theta_2}{1-\theta_2}\right)$ &
$\displaystyle\theta_1 = \frac{\theta_0}{\vartheta_x}\left(\frac{\theta_2}{-1\theta_2}\right)$ &
\multirow{2}{*}{$\displaystyle\vartheta_y\frac{x}{\vartheta_x}\left(1-\theta_2\left(1-\frac{x}{\vartheta_x}\right)\right)^{-1/\theta_2}$}\tabularnewline
 & 
 &
$\displaystyle\vartheta_y = \vartheta_x\left(\frac{1-\theta_2}{\theta_0}\right)^{1/\theta_2}$ &
$\displaystyle\theta_0 = (1-\theta_2)\left(\frac{\vartheta_y}{\vartheta_x}\right)^{-\theta_2}$
 & \tabularnewline
%\hline 
%---------------------------------------------------------------------------------------------------
% modelo de lactação de Wood
\multirow{3}{*}{14} &
\multirow{3}{*}{$\theta_0 x^{\theta_1}\exp\{-\theta_2 x\}$} &
$\displaystyle\vartheta_x = \theta_1/\theta_2$ &
$\displaystyle\theta_1 = \theta_2\vartheta_x$ &
\multirow{2}{*}{$\displaystyle \vartheta_y\left(\frac{x}{\vartheta_x}\right)^{\dot{\theta}_1}\exp\{\dot{\theta}_1(1-x/\vartheta_x)\} $}\tabularnewline
 & 
 &
$\displaystyle\vartheta_y = \theta_0(\theta_1/\theta_2)\exp\{-\theta_1\}$ &
$\displaystyle\theta_0 = \vartheta_y\left(\frac{1}{\vartheta_x}\right)^{\theta_1}\exp\{\theta_1\}$
 & \tabularnewline
 &
 &
$\displaystyle\vartheta_p = \theta_2^{(\theta_1+1)}$ &
$\displaystyle\theta_2 = \vartheta_p^{-1/(\theta_1+1)}$ &
 $\dot{\theta}_1\, : \, \dot{\theta}_1 - \vartheta_x \vartheta_p^{-1/(\dot{\theta}+1)}$ \tabularnewline
\hline 
%\end{tabular}
\end{tabular*}

% inclusões:
% Richards (tá comentado),
% Weibull,
% Langmuir, A*(B*x^C)/(1+B*x^C) Shabenberger exemplo 5.5
% Morgan-Mercer-Flodin,
% A*(exp(-B*t)/(1+D*exp(-B*t))^2) Ross2010,
% Brody,
% Mitscherlich, A*(1-exp(-k*(x-x0))) ou A+(E-A)*exp(-k*x)
% Shinozaki e Kira (1956), 1/(A+B*x), Shabenberger exemplo 5.8
% Holliday (1960), 1/(A+B*x+C*x^2), Shabenberger exemplo 5.8
% Arndt-Schulz law, Brain & Cousens (1989), D+(A-D+K*x)/(1+C*exp(B*log(x))), homesis em herbicida, Schabenberger figura 5.27
% Chapman-Richards, A(1-exp(B*x))^C, Schabenberger figura 5.36
% Von Bertalanffy


\end{small}
\end{sidewaystable}

\begin{sidewaystable}
 \caption{(cont.) \captabelas.}\label{tab:catalog3}
\begin{small}
 %\begin{tabular}{|c|c|c|c|c|}
%\begin{tabular*}{\textwidth}{@{\extracolsep{\fill}}|c|c|c|c|c|}
%\begin{tabular}{|C{0.4cm}|C{3.9cm}|C{3.5cm}|C{3.3cm}|C{5.5cm}|}
%\begin{tabular}{C{0.4cm}C{3.9cm}C{3.5cm}C{3.3cm}C{5.5cm}}
\begin{tabular*}{\textwidth}{@{\extracolsep{\fill}}ccccc}
\hline 
id &
Modelo original & $\vartheta = g(\boldsymbol{\theta})$ & $\theta_i = g^{-1}(\vartheta, \boldsymbol{\theta}_{-i})$ & Modelo reparametrizado\tabularnewline
\hline 
%---------------------------------------------------------------------------------------------------
% modelo logístico para fração de vida e dose máxima
\multirow{2}{*}{15} &
\multirow{2}{*}{$\displaystyle\frac{\theta_a}{1+\exp\{\theta_0+\theta_1 x\}}$} &
$\displaystyle\vartheta_q = \frac{1}{\theta_1}\left(\log\left(\frac{1-q}{q}\right)-\theta_0 \right)  $ &
$\displaystyle\theta_0 = \log\left(\frac{1-q}{q}\right)-\theta_1\vartheta_q$ &
\multirow{2}{*}{$\displaystyle\frac{\theta_a}{1+\left(\frac{1-q}{q}\right)\exp\left\{-4\vartheta_t(x-\vartheta_q) \right\}}$}\tabularnewline
 & 
 &
$\displaystyle\vartheta_t = -\frac{\theta_1}{4}$ &
$\displaystyle\theta_1 = -4\vartheta_t$
 & \tabularnewline
%\hline
%---------------------------------------------------------------------------------------------------
% modelo gompertz para fração de vida
 16 &
$\theta_a\exp\{-\exp\{\theta_0+\theta_1 x\}\}$ &
$\displaystyle\vartheta_q = \frac{\log(-\log(q))-\theta_0}{\theta_1}$ &
$\displaystyle\theta_1 = \frac{\log(-\log(q))-\theta_0}{\vartheta_x}$ &
$\displaystyle \theta_a\exp\{\log(q)\exp\{\theta_0(1-x/\vartheta_x)\}\}$\tabularnewline
%\hline
%---------------------------------------------------------------------------------------------------
% modelo de Richards
% 16 &
%$\displaystyle\frac{\theta_a}{(1+\exp\{\theta_0+\theta_1 x\})^{1/\theta_2}}$ &
%$\displaystyle\vartheta_q = \frac{\log((1-q^{\theta_2})/q^{\theta^2})-\theta_0}{\theta_1}$ &
%$\displaystyle\theta_1 = \frac{\log(\frac{1-q^{\theta_2}}{q^{\theta^2}})-\theta_0}{\vartheta_q}$ &
%$\displaystyle\frac{\theta_a}{(1+\exp\{\theta_0(1-x/\vartheta_q)+\log((1-q^{\theta_2})/q^{\theta_2})x/\vartheta_q\})^{1/\theta_2}}$\tabularnewline
%\hline
%---------------------------------------------------------------------------------------------------
% modelo de van Genuchten
\multirow{2}{*}{17} &
\multirow{2}{*}{$\displaystyle \theta_r+\frac{\theta_s-\theta_r}{(1+\exp\{\theta_a+x\}^{\theta_n})^{\theta_m}}$} &
$\displaystyle\vartheta_i = -\theta_a-\log(\theta_m)/\theta_n$ &
$\theta_a = -\vartheta_i-\log(\theta_m)/\theta_n $ &
\multirow{2}{*}{$\displaystyle \theta_r-\frac{\vartheta_s}{\theta_n}\frac{(1+1/\theta_m)^{\theta_m+1}}{(1+\exp\{\theta_n(x-\vartheta_i)\}/\theta_m)^{\theta_m}}$}\tabularnewline
 & 
 &
$\displaystyle\vartheta_s = -\frac{\theta_n(\theta_s-\theta_r)}{(1-1/\theta_m)^{\theta_m+1}}$ &
$\theta_s-\theta_r = -\frac{\vartheta_s}{\theta_n}(1+1/\theta_m)^{\theta_m+1} $
 & \tabularnewline
\hline 
%\end{tabular}
\end{tabular*}

\end{small}
\end{sidewaystable}

\newpage
\section{ESTIMAÇÃO}

Nessa seção será feita uma discussão sobre inferência sobre parâmetros
em modelos de regressão não linear. A inferência baseada em
verossimilhança será discutida, bem como inferência baseada na sua
aproximação quadrática. Por fim, uma revisão do método delta será
apresentada.

\subsection{Verossimilhança}

Seja $f(x, \boldsymbol{\theta})$ um modelo de regressão não linear
considerado para descrever a média de uma variável aleatória $Y$.
Considere que $Y$ tenha distribuição normal com variância constante
$\sigma^2$. Resumidamente, podemos escrever esse modelo como
\begin{align*}
 Y &\sim \text{Normal}(\mu(x), \sigma^2)\\
 \mu(x) &=  f(x,\boldsymbol{\theta}).
\end{align*}
A função de verossimilhança do modelo é dada por
\begin{equation}
 \text{L}(\boldsymbol{\theta}, \sigma^2) =
 \prod_{i=1}^{n} \phi(y_i, f(x_i, \boldsymbol\theta), \sigma^2),
\end{equation}
em que $\phi$ representa a função densidade da distribuição Normal. O
estimador de máxima verossimilhança são os valores
$(\hat{\boldsymbol\theta}, \hat\sigma^2)$ que tornam máximo o valor de
$\text{L}$. Para estimação, é conveniente trabalhar com o logaritmo da
função de verossimilhança
\begin{equation}
 \ell(\boldsymbol{\theta}, \sigma^2) =
 \log \text{L}(\boldsymbol{\theta}, \sigma^2).
\end{equation}

\subsection{Método delta}

O método delta é usado para aproximar a média e a variância de funções
não lineares de variáveis aleatórias. Dentre suas aplicações, uma das
mais comuns é relacionada à inferência sobre funções de parâmetros em
modelos de regressão, como a razão entre parâmetros, transformação de
um parâmetro, ou valor predito pelo modelo. Exemplos de funções de
parâmetros estão na terceira coluna das tabelas \ref{tab:catalog1} à
\ref{tab:catalog3}.

\newpage
\section{CONSIDERAÇÕES FINAIS}

A conclusão que se antecipa é que, uma vez que é possível
reparametrizar o modelo para o parâmetro de interesse, inferência
baseada na verossimilhança deve ser considerada, em segundo, sua
aproximação quadrática, visto a capacidade de modelagem permitida por
tais abordagens. Além do mais, as parametrizações devem ser avaliadas,
seja por meio de medidas de curvatura, gráficos de perfil ou
simulação, e deve ser escolhida àquela que tenha melhor compromisso
entre propriedades estatísticas e de interpretação.

\newpage
\addcontentsline{toc}{section}{\hspace*{\distnumber}REFERÊNCIAS}
\begin{center}
\section*{REFERÊNCIAS} 
\end{center}
