O efeito da desfolha sobre a qualidade e produtividade das culturas é
informação fundamental para definir estratégias de manejo, como
intensidade e frequência de pastejo e colheita até o estabelecimento
de níveis de dano econômico de forma a auxiliar decisões sobre o
controle de pragas desfolhadoras. Para a cultura do algodão, assim
como para outras tantas, a redução da produção pela desfolha pode ser
representada por uma função não linear monótona não
crescente. Diversos modelos podem satisfazer essa restrição, no
entanto, existe a preocupação de inferir sobre o nível de dano
econômico, $\vartheta_q$, pelo ajuste de um modelo. Dados de
produção-desfolha do algodoeiro em função do estágio fenológico são
considerados para inferir sobre o nível de dano econômico com os
seguintes objetivos: 1) propor uma parametrização de modelo que
representasse o parâmetro, 2) avaliar parametrizações alternativas por
meio de medidas de não linearidade, 3) aplicar inferência baseada em
verossimilhança, 4) selecionar um modelo para descrever a relação
entre produção e desfolha do algodoeiro em função do estágio
fenológico. O modelo reparametrizado apresentou menores medidas de não
linearidade nos estágios fenológicos com pronunciada relação não
linear. Nos restantes, as medidas de curvatura, as correlações dos
estimadores e os gráficos de perfil de verossimilhança indicaram que
um sub-modelo deveria ser considerado.\\
\newline
\noindent {Palavras-chave}: Interpretação de parâmetros. Verossimilhança. Método delta. Curvatura. \emph{Gossypium hirsutum}.
